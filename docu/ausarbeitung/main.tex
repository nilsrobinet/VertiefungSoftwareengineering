% Das ist die haupt Datei aus der die Ausarbeitung gerneriert wird. Zur besseren übersicht werden die einzellnen Kapitel als eigenen Dateien eingebunden
\documentclass[a4paper,12pt,numbers=noenddot]{scrreprt}
\usepackage[ngerman]{babel}
\usepackage[T1]{fontenc}
\usepackage{lmodern}
\usepackage{cite}
\usepackage{url}
\usepackage{hyperref}
\usepackage{siunitx}
\usepackage{multicol}
\usepackage{xcolor}
\usepackage[margin=10pt,font=small,labelfont=bf]{caption}
\sisetup{
	locale = DE ,
	per-mode = symbol
}
\usepackage{listings}
\definecolor{codegreen}{rgb}{0,0.6,0}
\definecolor{codegray}{rgb}{0.5,0.5,0.5}
\definecolor{codepurple}{rgb}{0.58,0,0.82}
\definecolor{backcolour}{rgb}{0.95,0.95,0.92}
\lstdefinestyle{mystyle}{
    backgroundcolor=\color{backcolour},   
    commentstyle=\color{codegreen},
    keywordstyle=\color{magenta},
    numberstyle=\tiny\color{codegray},
    stringstyle=\color{codepurple},
    basicstyle=\ttfamily\footnotesize,
    breakatwhitespace=false,         
    breaklines=true,                 
    captionpos=b,                    
    keepspaces=true,                 
    numbers=left,                    
    numbersep=5pt,                  
    showspaces=false,                
    showstringspaces=false,
    showtabs=false,                  
    tabsize=2
}
\lstset{style=mystyle}
\usepackage[printonlyused,withpage]{acronym}
\usepackage[fixlanguage]{babelbib}
\hypersetup{
% TODO: PDF setup
%    bookmarks=true,
    unicode=false,
    pdftoolbar=true,
    pdfmenubar=true,
    pdffitwindow=false,
    pdfstartview={FitH},
    pdftitle={},
    pdfauthor={Nils Robinet},
    pdfsubject={Thema},
    pdfcreator={Nils Robinet},
    pdfproducer={},
    pdfkeywords={} {} {} {},
    pdfnewwindow=true,
    colorlinks=true,
    linkcolor=black,
    citecolor=black,
    filecolor=black,
    urlcolor=black,
%    pagebackref=true
}

\title{Ausarbeitung zum Modul Vertiefung Softwareengineering}
\author{Nils Robinet}

\begin{document}

\maketitle

\tableofcontents

\pagenumbering{arabic}
\setcounter{page}{1}

\chapter*{Abkürzungsverzeichnis}
\begin{acronym}
    \acro{CI/CD}{Continues Integration/Continues Deploy}
    \acro{VM}{Virtual Machine}
    \acro{LTS}{Long Term Service}
\end{acronym}

\chapter{Einleitung}

In dieser Ausarbeitung soll das Abschlussprojekt im Fach \glqq Vertiefung Softwareengineering\grqq{} dokumentiert werden. Im Abschlussprojekt sollte eine \ac{CI/CD} Pipeline aufgesetzt werden und ein Projekt angelegt werden das Beispielhaft über die Pipeline gebaut, gtestet und deployt wird. Als Projekt wurde sich für eine Implementation des Eigengesichtsalogrithmus entschieden. 

Im ersten Kapitel wird beschrieben, wie die Entwicklungsumgebung aufegbaut ist. Im zweiten Kapitel wird kurz auf die git Entwicklungsstrategie eingegangen, mit der das Projekt einwickelt wird. Im Folgenden wird kurz auf die Anwendung, die in dem \ac{CI/CD}-Projekt bebaut wird, eingegangen.
\chapter{Tools}
\label{chap:tools}

Um das Projekt zu bauen werden einige Tools benötigt. Diese sollen hier aufgelistet und ihre Verwendung erklärt werden.
Eine vollständige Liste der Tools die zum bauen des Projekts verwendet werden, kann den Dateien \glqq Dockerfile \grqq{} und \glqq requirements.txt \grqq{} entnommen werden.

\begin{description}
    \item[pdflatex] ist ein Typesetter der verwendet wird um aus laTeX markup PDF Dokumente zu erzeugen. Es wird hier verwerndet, um dieses PDF Dokument zu erstellen. \cite{MANLATEX}.
    \item[Marp] ist eine Umgebung, um aus Markdown Präsentationen zu generieren. Es wird verwendet, um für die Päsentation zur Ausarbeitung ein PDF zu erstellen \cite{MARP2024}.
    \item[Jenkins] ist ein System zur Umsetzung von \glqq contiues integratio \grqq{} und \glqq continues deployment\grqq{} von Software \cite{JENKINS2024}.
    \item[gcc / g++] Die GNU Compiler Collection ist eine Sammlung von Kompilern, insbesondere für die Programmiersprachen der C Familie\cite{GCC2024}. Sie wird in diesem Projekt zum übersetzen von C++ Code genutzt.
    \item[CMake / Make] CMake ist ein Meta Build Generator, der genutzt wird um die eigentlichen Build Generatoren wie Make zu konfigurieren. Make wird genutzt um das eigentliche Bauen und Linken von Programmen zu vereinfachen\cite{CMAKE2024}.
    \item[cython] Cython ist ein Werkzeug, das es ermöglicht C/C++ Erweiterungen für Python einfach zu programmieren. Cython kompiliert Python und die Cython-Programmiersprache in C/C++ Code, der mit allen gängigen C/C++-Kompilern kompiliert werden kann. Herauskommen CPython kompatible Shared-Library-Objekte\cite{behnel2011}.
    \item[Docker] Docker ist eine Container-Engine, die es ermöglicht Anwendungen zu Isolieren und ihnen eine stabile Laufzeitumgebung zu garantieren. In einen Container kann eine Anwendung mit all ihren Abhängigkeiten gepackt werden. So ist ein Einfacher Transport und Auslieferung möglich\cite{merkel2014}.
\end{description}

\chapter{Entwicklungsumgebung}

Für das Projekt sollte eine \acf{CI/CD} Pipline angelegt werden. Zu diesem Zweck wird ein Build Server in einer \acf{VM} gestartet.

\section{Werkzeuge und Anwendungen}

Der Code des projektes wird mit der Versionsverwaltungssoftware GIT gemanaged (mehr zur git Strategie in chap TODO). Um das remote repository zu managen wird auf Github gesetzt. Die \ac{CI/CD} Pipeline wird durch den open source Build-Server Jenkins gemanaged. Die Build-Toolchain wird in einem Docker-Container bereit gestellt. Der Build-Server läuft in einer \ac{VM}. Als Virtualisierungslösung wurde QEMU/KVM ausgeählt. Die Builds werden mittels CMake und Bash-Scipten Konfiguriert.

\section{Build-Server Setup}

Als Guest-Beriebsystem wurde sich für Arch Linux entschieden, da von diesem Betriebsystem bereits fertig vorgebaute \ac{VM} images bereit gestellt werden (\url{https://gitlab.archlinux.org/archlinux/arch-boxes/-/packages}). Unter anderem wird ein qcow2 image das in der Virtualisierungsumgebung QEMU eingesetzt werden kann bereit gestellt.
In der \ac{VM} wird ein Jenkins build server installiert, der die CI/CD pipline bedient.

\subsection{Installation}

Die Build-Server \ac{VM} wurde auf ein Arch Linux System erstellt, einzellne Schritte, insbesondere wenn eine Packetverwaltungssoftware eingesetzt wird können sich zwischen den verschiedenen Betriebsystemen unterscheiden.
Zur erstellung und installation wurden folgende Schritte durch geführt:

\begin{enumerate}
    \item Datenträgerabbild herunter laden:\\
        \lstinline[language=sh]!wget -O build_server.qcow2 https://gitlab.archlinux.org/archlinux/arch-boxes/-/package_files/6852/download!
    \item Virtuelle Machine Starten:\\
    \lstinline[language=sh]!qemu-system-x86_64 -m 4G -smp 4 -enable-kvm\! \\
    \lstinline[language=sh]!                   -nic user,hostfwd=tcp::60022-:22,hostfwd=tcp::8090-:8090\ !\\
    \lstinline[language=sh]!                   build_server.qcow2!\\
    Der Benutzername und das Standartpasswort lauten: arch/arch; Man kann sich nun mittels ssh mit der \ac{VM} verbinden:\\
    \lstinline[language=sh]!ssh arch@localhost -p 60022!\\
    \item Jenkins installieren:\\
    \lstinline[language=sh]!pacman -Syu !\\
    \lstinline[language=sh]!pacman -S fontconfig!\\
    \lstinline[language=sh]!pacman -S freetype2!\\
    \lstinline[language=sh]!pacman -S jenkins!\\
    \lstinline[language=sh]!pacman -S docker!\\
    \lstinline[language=sh]!pacman -S git!\\
    \lstinline[language=sh]!systemctl enable jenkins!\\
    \lstinline[language=sh]!systemctl enable docker!\\
    \lstinline[language=sh]!systemctl start jenkins!\\
    \lstinline[language=sh]!systemctl start docker!\\
    \lstinline[language=sh]!sudo usermod -aG docker jenkins!\\
    \item Nach der installation kann über den Webrowser auf dem Host Beriebsystem unter der Addresse http://localhost:8090 auf die jenkins installation zu gegriffen werden. Im initialen Setup-dialog muss ausgewählt werden ob dieStandar-Plugins installiert werden sollen, hier sollte mit ja geantwortet werden, und ein Admin Benutzer angelegt werden. Danach landet man im Haupmenu der Jenkins installation.

Zusätzlich wird noch das Plugin "Docker Pipeline" benötigt, das es ermöglicht in docker images zu bauen.
\end{enumerate}

\subsection{Projekt einrichtung}

Unter dem Punkt "new job" wird ein neuer Multibranch Pipeline Job angelegt und Konfiguriert. Als Branch Source wird das github Repo dieses Projekts angegeben. Damit jenkins auf das Projekt zu greifen kann müsen noch Github credentials eingrtragen werden. 

Damit Jenkins Änderungen zeitnah erkennt wir ein periodischer Trigger mit einem Intervall von einer Minute eingestellt. Danach ist das Projekt fertig konfiguriert und kann benutzt werden.

Die Build und Deployment stages werden durch das im Repositroy eingecheckte Jenkinsfile konfgiuriert. Der build findet in einem Docker-Container statt der durch ein Dockerfile, das ebenfalls eingecheckt ist, erstellt wird. Das docker image wird durch Jenkins gemanaged, Jenkins kümmert sich auch darum das Dockerimage beiÄnderungen am Dockerfile neu zu generieren.


\end{document}
