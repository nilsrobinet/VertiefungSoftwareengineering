\chapter{Tools}

Um das projekt zu bauen werden einige Tools benötigt. Diese sollen hier aufgelistet und ihre Verwendung erklärt werden.

\begin{description}
    \item[pdflatex] ist ein typesetter der verwendet wird um aus laTeX markup PDF Dokumente zu erzeugen. Es wird hier verwerndet, um dieses PDF Dokument zu erstellen.
    \item[Marp] ist eine  umgebung um aus Markdown Präsentationen zu generieren. Es wird verwendet um für die Päsentation zur ausarbeitung ein PDF zu erstellen.
    \item[Jenkins] ist ein System zur Umsetzung von "contiues integration" und "continues deployment" von Software.
    \item[gcc / g++] Die GNU compiler collection ist eine Sammlung von Kompilern, insbesondere für die Programmiersprachen der C Familie. Sie wird in diesem Projekt zum übersetzen von C++ Code genutzt.
    \item[CMake / Make] CMake ist ein meta build generator ger genutzt wird um die eigentlichen build generatoren wie Make zu konfigurieren. Make wird genutzt um das eigentliche bauen und linken von Programmen zu vereinfachen.
\end{description}
