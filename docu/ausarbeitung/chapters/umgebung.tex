\chapter{Entwicklungsumgebung}

F�r das Projekt sollte eine \acf{CI/CD} Pipline angelegt werden. Zu diesem Zweck wird ein Build Server in einer \acf{VM} gestartet.

\section{Werkzeuge und Anwendungen}

Der Code des projektes wird mit der Versionsverwaltungssoftware GIT gemanaged (mehr zur git Strategie in chap TODO). Um das remote repository zu managen wird auf Github gesetzt. Die \ac{CI/CD} Pipeline wird durch den open source Build-Server Jenkins gemanaged. Die Build-Toolchain wird in einem Docker-Container bereit gestellt. Der Build-Server läuft in einer \ac{VM}. Als Virtualisierungslösung wurde QEMU/KVM ausgew�hlt. Die Builds werden mittels CMake und Bash-Scipten Konfiguriert.

\section{Build-Server Setup}

Als Guest-Beriebsystem wurde sich f�r Arch Linux entschieden, da von diesem Betriebsystem bereits fertig vorgebaute \ac{VM} images bereit gestellt werden (\url{https://gitlab.archlinux.org/archlinux/arch-boxes/-/packages}). Unter anderem wird ein qcow2 image das in der Virtualisierungsumgebung QEMU eingesetzt werden kann bereit gestellt.
In der \ac{VM} wird ein Jenkins build server installiert, der die CI/CD pipline bedient.

\subsection{Installation}

Die Build-Server \ac{VM} wurde auf ein Arch Linux System erstellt, einzellne Schritte, insbesondere wenn eine Packetverwaltungssoftware eingesetzt wird können sich zwischen den verschiedenen Betriebsystemen unterscheiden.
Zur erstellung und installation wurden folgende Schritte durch gef�hrt:

\begin{enumerate}
    \item Datenträgerabbild herunter laden:\\
        \lstinline[language=sh]!wget -O build_server.qcow2 https://gitlab.archlinux.org/archlinux/arch-boxes/-/package_files/6852/download!
    \item Virtuelle Machine Starten:\\
    \lstinline[language=sh]!qemu-system-x86_64 -m 4G -smp 4 -enable-kvm\! \\
    \lstinline[language=sh]!                   -nic user,hostfwd=tcp::60022-:22,hostfwd=tcp::8090-:8090\ !\\
    \lstinline[language=sh]!                   build_server.qcow2!\\
    Der Benutzername und das Standartpasswort lauten: arch/arch; Man kann sich nun mittels ssh mit der \ac{VM} verbinden:\\
    \lstinline[language=sh]!ssh arch@localhost -p 60022!\\
    \item Jenkins installieren:\\
    \lstinline[language=sh]!pacman -Syu !\\
    \lstinline[language=sh]!pacman -S fontconfig!\\
    \lstinline[language=sh]!pacman -S freetype2!\\
    \lstinline[language=sh]!pacman -S jenkins!\\
    \lstinline[language=sh]!pacman -S docker!\\
    \lstinline[language=sh]!pacman -S git!\\
    \lstinline[language=sh]!systemctl enable jenkins!\\
    \lstinline[language=sh]!systemctl enable docker!\\
    \lstinline[language=sh]!systemctl start jenkins!\\
    \lstinline[language=sh]!systemctl start docker!\\
    \lstinline[language=sh]!sudo usermod -aG docker jenkins!\\
    \item Nach der installation kann �ber den Webserver auf dem Host Beriebsystem unter der Addresse http://localhost:8090 auf die jenkins installation zu gegriffen werden. Im initialen Setup-dialog muss ausgewählt werden ob dieStandar-Plugins installiert werden sollen, hier sollte mit ja geantwortet werden, und ein Admin Benutzer angelegt werden. Danach landet man im Haupmenu der Jenkins installation.

Zus�tzlich wird noch das Plugin "Docker Pipeline" ben�tigt, das es erm�glicht in docker images zu bauen.
\end{enumerate}

\subsection{Projekt einrichtung}

Unter dem Punkt "new job" wird ein neuer Multibranch Pipeline Job angelegt und Konfiguriert. Als Branch Source wird das github Repo dieses Projekts angegeben. Damit jenkins auf das Projekt zu greifen kann m�ssen noch Github credentials eingrtragen werden. 

Damit Jenkins �nderungen zeitnah erkennt wir ein periodischer Trigger mit einem Intervall von einer Minute eingestellt. Danach ist das Projekt fertig konfiguriert und kann benutzt werden.

Die Build und Deployment stages werden durch das im Repositroy eingecheckte Jenkinsfile konfgiuriert. Der build findet in einem Docker-Container statt der durch ein Dockerfile, das ebenfalls eingecheckt ist, erstellt wird. Das docker image wird durch Jenkins gemanaged, jeknins k�mmert sich auch darum das Dockerimage bei �nderungen am Dockerfile neu zu generieren.
