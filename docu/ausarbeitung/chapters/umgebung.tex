\chapter{Entwicklungsumgebung}

Für das Projekt sollte eine \acf{CI/CD} Pipline angelegt werden. Zu diesem Zweck wird ein Build Server in einer \acf{VM} gestartet.

\section{\ac{VM} installation}

Als \ac{VM} wurde sich für Ubuntu 24.04 Server entschieden, da diese Version eine \acf{LTS} Version ist, kann sie für einen längeren Zeitraum verwendet werden ohne das man ein Betriebsystem Upgrade durchführen muss, bei dem es zu Soft- und Hardware inkompatibilitäten kommen kann. Die Server Version wurde ausgewählt, da für die Verwendung als Build Server keine grafische Benutzeroberfläche benötigt wird. Als virtualisierungs Lösung wird die open source Software QEMU eingesetzt.

\subsection{QEMU Setup}

Für den einsatz von QEMU unter Windows wird MYSYS2 benötigt. MYSYS2 ist eine Software Distributionsplatform basierend auf Cygwin. Cygwin ist ein Projekt das viele Tools aus dem GNU/LINUX Umfeld für Windows portiert und so ein Linux ähnliches arbeiten ermöglicht. Unteranderem bringt MYSYS2 den aus der Linux Distribution Arch bekannten Packetmanager pacman mit die hier genutzt werden kann um QEMU zu installieren.
 
Auf den üblichen Linux Distibutionen lässt sich QEMU einfach über die Packetverwaltung installieren.

