\chapter{Entwicklungsumgebung}

Für das Projekt sollte eine \acf{CI/CD} Pipline angelegt werden. Zu diesem Zweck wird ein Build Server in einer \acf{VM} gestartet.

\section{Werkzeuge und Anwendungen}

Der Code des projektes wird mit der Versionsverwaltungssoftware GIT gemanaged (mehr zur git Strategie in chap TODO). Um das remote repository zu managen wird auf Github gesetzt. Die \ac{CI/CD} Pipeline wird durch den open source Build-Server Jenkins gemanaged. Die Build-Toolchain wird in einem Docker-Container bereit gestellt. Der Build-Server läuft in einer \ac{VM}. Als Virtualisierungslösung wurde QEMU/KVM ausgewählt. Die Builds werden mittels CMake und Bash-Scipten Konfiguriert.

\section{Build-Server Setup}

Als Guest-Beriebsystem wurde sich für Arch Linux entschieden, da von diesem Betriebsystem bereits fertig vorgebaute \ac{VM} images bereit gestellt werden (\url{https://gitlab.archlinux.org/archlinux/arch-boxes/-/packages}). Unter anderem wird ein qcow2 image das in der Virtualisierungsumgebung QEMU eingesetzt werden kann bereit gestellt.
In der \ac{VM} wird ein Jenkins build server installiert, der die CI/CD pipline bedient.

\subsection{Installation}

Die Build-Server \ac{VM} wurde auf ein Arch Linux System erstellt, einzellne Schritte, insbesondere wenn eine Packetverwaltungssoftware eingesetzt wird können sich zwischen den verschiedenen Betriebsystemen unterscheiden.
Zur erstellung und installation wurden folgende Schritte durch geführt:

\begin{enumerate}
    \item Datenträgerabbild herunter laden:\\
        \lstinline[language=sh]!wget -O build_server.qcow2 https://gitlab.archlinux.org/archlinux/arch-boxes/-/package_files/6852/download!
    \item Virtuelle Machine Starten:\\
    \lstinline[language=sh]!qemu-system-x86_64 -m 4G -smp 4 -enable-kvm\! \\
    \lstinline[language=sh]!                   -nic user,hostfwd=tcp::60022-:22,hostfwd=tcp::8090-:8090\ !\\
    \lstinline[language=sh]!                   build_server.qcow2!\\
    Der Benutzername und das Standartpasswort lauten: arch/arch; Man kann sich nun mittels ssh mit der \ac{VM} verbinden:\\
    \lstinline[language=sh]!ssh arch@localhost -p 60022!\\
    \item Jenkins installieren:\\
    \lstinline[language=sh]!pacman -Syu !\\
    \lstinline[language=sh]!pacman -S fontconfig!\\
    \lstinline[language=sh]!pacman -S freetype2!\\
    \lstinline[language=sh]!pacman -S jenkins!\\
    \lstinline[language=sh]!systemctl enable jenkins!\\
    \item Nach der installation kann über den Webserver auf dem Host Beriebsystem unter der Addresse http://localhost:8090 auf die jenkins installation zu gegriffen werden. Im initialen Setupdialog muss ausgewählt werden ob die standard plugins installiert werden sollen und ein Admin Benutzer angelegt werden. Danach landet man im Haupmenu der Jenkins installation.
\end{enumerate}

\subsection{Projekt einrichtung}

Unter dem Punkt "new job" wird ein neuer Multibranch Pipeline Job angelegt. Der Job wird durch das Jenkinsfile, das im Repository eingecheckt ist konfiguriert
